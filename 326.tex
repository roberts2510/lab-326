% "Станет проще"

\documentclass[a4paper,12pt]{article} % тип документа

% report, book

% Рисунки
\usepackage{graphicx}
\usepackage{wrapfig}
\usepackage{hyperref}
\usepackage[rgb]{xcolor}
\pagestyle{plain}



%  Русский язык

\usepackage[T2A]{fontenc}			% кодировка
\usepackage[utf8]{inputenc}			% кодировка исходного текста
\usepackage[english,russian]{babel}	% локализация и переносы


% Математика
\usepackage{amsmath,amsfonts,amssymb,amsthm,mathtools} 


\usepackage{wasysym}

%Заговолок
\author{Сафиуллин Роберт	785 группа}
\title{"Электромагнитный поезд"}





\begin{document} % начало документа

\maketitle


\newpage
\section{Устройство}




Рассчитаем силы, действующие в установке: \\
Сила сцепления магнитов с d=10 mm, h=5 mm: $ 2.66 kg\Rightarrow 26.1*10^5$ дин $\Rightarrow$ магнитный момент неодимового магнита найдем по формуле: $P_m= \sqrt{\frac{F*d^4}{6}}=164.9 cm^2$ \\
Поле соленоида $B=\frac{2\pi In}{c} (cos\beta-cos\alpha)$\\
Учитывая, что n=3.5 витка/cm, длина соленоида l=4.2cm, а радиус r=0.85 cm, найдем поле у торца: \\
$B_1$=								\\
А также в центре неодимового магнита($\cos\beta=\frac{r}{l+\frac{h}{2}}$) \\
$B_2$=















\section{Цель работы:}
 изучение вольт-ампернйо характеристики нормального тлеющего разряда; исследование релаксационного генератора на стабилитроне.
\\
стабилитрон СГ-2, амперметр, вольтметр, магазин сопротивлений, магазин емкостей, источник питания, осциллограф, генератор звуковой частоты
 
\section{Экспериментальная установка:}
\begin{center}


\end{center}

\section{Ход работы}

\textbf{I: Характеристика стабилитрона}\\
1) Собрали первую схему
2) Сняли вольт-амперную характеристику. Результаты занессли в таблицу: \\
\begin{tabular}{|c|c|c|c|c|c|}
\hline 
V, В & I, *0.25 mA & $V_z$, В & $I_z$, *0.25 mA & $V_g$, В & $I_g$, *0.25 mA \\ 
\hline 
87 & 11 & 96.9 & 18 & 75 & 2.3 \\ 
\hline 
84.7 & 10 & 97 & 18.5 & 75.7 & 2.2 \\ 
\hline 
82 & 8 & 97.03 & 19.3 & • & • \\ 
\hline 
81.4 & 7 & • & • & • & • \\ 
\hline 
79.7 & 6 & • & • & • & • \\ 
\hline 
78.2 & 5 & • & • & • & • \\ 
\hline 
77.2 & 4 & • & • & • & • \\ 
\hline 
75.9 & 2.5 & • & • & • & • \\ 
\hline 
97 & 14 & • & • & • & • \\ 
\hline 
93 & 15 & • & • & • & • \\ 
\hline 
94 & 16 & • & • & • & • \\ 
\hline 
89 & 12 & • & • & • & • \\ 
\hline 
85 & 10 & • & • & • & • \\ 
\hline 
84 & 9 & • & • & • & • \\ 
\hline 
\end{tabular} 



































1) Снимем зависимость отклонения зайчика х от сопротивления магазина, при постоянном положении делителя $\frac{R_1}{R_2}=\frac{1}{1000}$. Результаты занесем в таблицу, рассчитав ток по формуле: I=$\frac{R_{1}U_{0}}{R_{2}(R+R_{0})	}$, $R_0$=610 $\Omega$:\\
\begin{tabular}{|c|c|c|}
\hline 
x, mm & R, к$\Omega$ & I, A*$10^{-9}$ \\ 
\hline 
12 & 50 & 2.96 \\ 
\hline 
25 & 30 & 4.9 \\ 
\hline 
51 & 20 & 7.3 \\ 
\hline 
71 & 17 & 8.5 \\ 
\hline 
83 & 16 & 9.03 \\ 
\hline 
98 & 15 & 9.6 \\ 
\hline 
118 & 14 & 10.2 \\ 
\hline 
151 & 13 & 11 \\ 
\hline 
209 & 12 & 11.9 \\ 
\hline 
\end{tabular}
\\
Построим график I(x) по этой таблице: \\
\begin{flushleft}




\end{flushleft}

Динамическую постоянную найдем по формулe: $C_I=2ak$, где а-расстояние до гальванометра=132см\\
$C_I=1.14*10^{-9}\frac{A*m}{mm}$\\
\textbf{II Определение критического сопротивления}\\
2) Разомкнем К2 и пронаблюдаем свободные колебания рамки. Измерим  декремент затухания разомкнутого гальванометра:\begin{tabular}{|c|c|c|}
\hline 
$x_1$,  mm & $x_2$, mm & $T_0$, c \\ 
\hline 
174 & 146 & 2.5  \\ 
\hline 
\end{tabular} 
\\
$\Theta_0=0.18$\\
3) Снова замкнем ключ К2 и подберем наибольшее значение магазина, при котором зайчик не переходит нулевое значение при размыкании К3. Это сопротивление близко к критическому: $R_{kr}=3800 \Omega$\\
2)Проведем измерения  двух последовательных
отклонений зайчика и рассчитаем логарифмический декремент затухания. Результаты занесем в таблицу:\\
\begin{tabular}{|c|c|c|c|c|c|}
\hline 
x1, mm & x2, mm & R, $\Omega$ & $\Theta$ & 1/$\Theta^{2}$ & $((R+R_0)^2)*10^{6}$ , $\Omega^2$\\ 
\hline 
67 & 9 & 12000 & 2 & 0.25 & 159 \\ 
\hline 
89 & 15 & 14000 & 1.78 & 0.72 & 213.5 \\ 
\hline 
86 & 17 & 16000 & 1.62 & 0.38 & 276 \\ 
\hline 
82 & 20 & 18000 & 1.41 & 0.5 & 346 \\ 
\hline 
121 & 32 & 20000 & 1.33 & 0.56 & 425 \\ 
\hline 
119 & 32 & 22000 & 1.31 & 0.58 & 511 \\ 
\hline 
117 & 35 & 23000 & 1.2 & 0.69 & 557.5 \\ 
\hline 
110 & 38 & 24000 & 1.06 & 0.9 & 605.6 \\ 
\hline 

\end{tabular} 
\\


По данным таблицы построим график зависимости $(1/\Theta^{2})((R+R_0)^2)$ \\
\begin{flushleft}

\end{flushleft}

Отсюда по формуле $R_{kr}=\frac{1}{2\pi}\sqrt{{\frac{\Delta X }{\Delta Y}}}-R_0$ получим, что $R_{kr}=3645 \Omega$


\textbf{III Баллистический режим}\\
5) Соберем схему : рис.3\\
6)Измерим первый отброс зайчика в режиме свободных колебаний , при установим делиттель в положение $\frac{R_1}{R_2}=\frac{1}{10}$\\
7) Снимем зависимость первого отброса зайчика $l_{maх}$ от сопротивления магазина. Результаты занесем в таблицу:\\
\begin{tabular}{|c|c|c|}
\hline 
$l_{max}$, mm & R, $\Omega$ & $ 1/(R+R_0) *10^{-6}, \Omega^{-1}$ \\ 
\hline 
260 & $\infty$ & 0 \\ 
\hline 
240 & 50 & 19 \\ 
\hline 
231 & 45 & 27 \\ 
\hline 
226 & 40 & 25 \\ 
\hline 
215 & 30 & 33 \\ 
\hline 
210 & 25 & 39 \\ 
\hline 
196 & 20 & 48 \\ 
\hline 
181 & 15 & 64 \\ 
\hline 
171 & 13 & 73 \\ 
\hline 
165 & 12 & 79 \\ 
\hline 
161 & 11 & 86 \\ 
\hline 
150 & 9 & 104 \\ 
\hline 
142 & 8 & 116 \\ 
\hline 
132 & 7 & 131 \\ 
\hline 
121 & 6 & 151 \\ 
\hline 
90 & 4 & 217 \\ 
\hline 
80 & 3 & 277 \\ 
\hline 
65 & 2.5 & 321 \\ 
\hline 
225 & 35 & 28 \\ 
\hline 
\end{tabular} 
\\
8) По этим данным построим график $l_{max}(1/(R+R_0))$\\
\begin{flushleft}

\end{flushleft}


Отсюда, зная что $R_{kr}$ соответствует значению l=95.5 mm, получим $R_{kr}$=3895 $\Omega$
\\
Все 3 полученных разными способами значения $R_{kr}$ лежат в диапозоне 3.6-3.9 k$\Omega$\\
9) Посчитаем баллистическую постоянную в критическом рижеим по формуле: \\
\begin{center}

$C_{kr}=2a\frac{R_1}{R_2}\frac{CU_0}{L_{max_{kr}}}$, где C=2 мкФ. \\
 $C_{kr}=8.3*10^{-9} \frac{K*m}{mm}$


Отношение баллистических постоянных: $\frac{C_{kr}}{C_I}=7.28$\\
Время релаксации: t=$CR_0=0.00122<<2.5=T_0$
\end{center}





















































\end{document} % конец документа
